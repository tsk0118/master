\documentclass[a5j,10.5pt]{jreport}


\usepackage[dvipdfmx]{graphicx}
\usepackage{latexsym}
\setcounter{secnumdepth}{5}

\usepackage{bmpsize}
\usepackage{url}
\usepackage{comment}
\usepackage[top=15truemm,bottom=30truemm,left=25truemm,right=25truemm]{geometry}

\renewcommand{\bibname}{参考文献}


\begin{document}
\begin{flushright}
{\Huge 21-xx}
\end{flushright}
\markright{表示するヘッダー情報}

\begin{center}
{\huge 修士論文}\\

\vspace*{1cm}
{\LARGE 深層生成モデルを用いた少数データからのデータ生成に関する研究}\\
\vspace*{2cm}
{\LARGE  2022年3月}\\
\vspace*{1cm}
{\LARGE  田崎裕也}\\
\vspace*{3cm}
{\Large  宇都宮大学大学院 修士課程}\\
\vspace*{1cm}
{\Large 地域創生科学研究科}\\
\vspace*{1cm}
{\Large  工農総合科学専攻} \\
\vspace*{1cm}
{\Large 情報電気電子システム工学プログラム}\\
\end{center}
\thispagestyle{empty}
\clearpage
\addtocounter{page}{-1}
\newpage
\begin{center}
    {\Huge 内容概要}\\
\end{center}
あとでかく
\newpage
\begin{center}
    {\Huge English Title}\\
    あとでかく
\end{center}
\newpage
\tableofcontents
\clearpage
\chapter{}
\section{節の名前}
\subsection{小節の名前}
\subsubsection{小小節の名前}
\paragraph{段落の名前}
\subparagraph{小段落の名前}
\chapter{部の名前2}
\section{節の名前2}
\begin{thebibliography}{数字}
  \bibitem{キー1} 参考文献の名前・著者1
  \bibitem{キー2} 参考文献の名前・著者2
  ・・・
  \bibitem{キーN} 参考文献の名前・著者N
\end{thebibliography}
\end{document}

